\documentclass{article}

\usepackage[top=1in, bottom=1in, left=1in, right=1in]{geometry}

\begin{document}

\title{Project \#3 Single Server Key-Value Store} 
\author{Hamel Ajay Kothari (de), Maliena Guy (dd), Bryan Cote-Chang (ed), EunSeon Son (et)}
\date{Section 106, TA: Kevin Klues}
\maketitle

\section*{Task 1: Message Parsing}
\subsection*{Correctness Constraints}
\begin{itemize}
\item An error has to be raised in any case of messages cannot be built and processed properly.
\end{itemize}

\subsection*{Declarations}
There are no declarations necessary for this section.

\subsection*{Description}

We will define a helper method which will make bulletproofing of message types much easier:
\begin{verbatim}
boolean validMsgType(String msgType):
    if (msgType is one of “getreq”, “putreq”, “delreq”, or “resp”)
        return true
    return false
\end{verbatim}

The two String based constructors are fairly simple to implement:
\begin{verbatim}
KVMessage(String msgType) {
    if (validMsgType(msgType))
        this.msgType = msgType;
    } else {
        throw new KVException("resp", "Message format incorrect");
    }
}
KVMessage(String msgType, String message) {
    if (validMsgType(msgType)) {
        this.msgType = msgType;
        this.message = message;
    } else {
        throw new KVException("resp", "Message format incorrect");
    }
}
\end{verbatim}

KVMessage() takes in a socket is from the network connection that is then parsed. Using the DocumentBuilder class from the Java XML libraries, we can parse the input and obtain the msgType.
\begin{verbatim}
KVMessage(Socket sock) {
    try {
        DocumentBuilderFactory dbFactory = DocumentBuilderFactory.newInstance();
        DocumentBuilder dBuilder = dbFactory.newDocumentBuilder();
        Document doc = dBuilder.parse(new NoCloseInputStream(sock.getInputStream()));
        doc.getDocumentElement().normalize();
        NodeList nList = doc.getElementsByTagName("KVMessage");
        Node nNode = nList.item(0);
        this.msgType = nNode.getAttribute("type");
        if (this.msgType == "getreq" || this.msgType == "delreq") {
            this.key = ((Element)nNode).getElementsByTagName("Key").item(0).getTextContent();
        } else if (this.msgType ==  "putreq") {
            this.key = ((Element)nNode).getElementsByTagName("Key").item(0).getTextContent();
            this.value = ((Element)nNode).getElementsByTagName("Value").item(0).getTextContent();
        } else if (this.msgType == "resp") {
            if (getresponse) {
                this.key = ((Element)nNode).getElementsByTagName("Key").item(0).getTextContent();
                this.value = ((Element)nNode).getElementsByTagName("Value").item(0).getTextContent();
            } else {
                this.message = ((Element)nNode).getElementsByTagName("Message").item(0).getTextContent();
            }
        } else {
            throw new KVException(new KVMessage("resp", "Unknwon Error: msgType is unknown"));
        }
    } catch (SAXException e) {
        throw new KVException(new KVMessage("resp", "XML Error: Received unparseable message"));
    } catch (IOException e) {
        throw new KVException(new KVMessage("resp", "Network Error: Could not receive data"));
    } catch (ParserConfigurationException e) {
        throw new KVException(new KVMessage("resp", "XML Error: Received unparseable message"));
    }
    if (this.key.length() > 256) {
        throw new KVException(new KVMessage("resp", "Oversized key"));
    }
    if (this.value.length() > 262144) {
        throw new KVException(new KVMessage("resp", "Oversized value"));
    }
}
\end{verbatim}

toXML() generates XML string for messages. It first checks msgType and tries to output the correct format of XML string. When there are not enough data or invalid data, it throws an exception.
\begin{verbatim}
toXML() {
    try {
        DocumentBuilderFactory dbFactory = DocumentBuilderFactory.newInstance();
        DocumentBuilder dBuilder = dbFactory.newDocumentBuilder();
        Document doc = dBuilder.parse(new NoCloseInputStream(sock.getInputStream()));
        set doc’s root “KVMessage”
        if (key > 256 byte) {
            throw new KVException(new KVMessage("resp", "Oversized key"));
        }
        if (value > 262144 byte) {
            throw new KVException(new KVMessage("resp", "Oversized value"));
        }
        if (msgType is “getreq”) {
            add type and key as childNode;
            return generated get request;
        } else if (msgType is “putreq”) {
            add type, key and value as childNode;
            return generated put request;
        } else if (msgType is “delreq”) {
            add type and key as childNode;
            return generated delete request;
        } else if (msgType is “resp”) {
            if (put or delete resp) {
                add type and msg as childNode;
            } else {
                add type, key and value as childNode;
            }
            return generated response;
        } else {
            throw an exception;    //invalid msgType
        }
    } catch (KVException e) {
        //not enough data is available to generate a valid KV XML message
    }
}
\end{verbatim}

sendMessage() takes in a socket and calls toXML() to generate the XML message. then writes
it on the socket using java built-in printWriter.
\begin{verbatim}
sendMessage(Socket sock) {
    try {
        xml = toXML();
        pw = new printWriter(NoCloseOutputStream(sock));
        pw prints the message and writes on the socket;
        socket.shutdownOutput();
        pw.close();
    } catch (IOException e) {
        throw new KVException(new KVMessage("resp", "Network Error: Could not receive data"));
    }
}
\end{verbatim}

\subsection*{Testing Plan}

In order to unit test our KVMessage we made 8 test cases, 3 for creating and generating XML for messages
and 5 for reading a message from a socket. Our cases are as follows:
\begin{enumerate}
\item To test that we could properly generate the right information for a put request, we create a KVMessage of type
"putreq" and set the key and value to test values. Then we called \textit{toXML()} on the message and asserted that the
result contained a "<Key>" tag with the test key and a value tag with the test value.
\item To test that we could properly for a get request, we did something similar to what was said above but instead of
type "putreq" we used type "getreq", set only the key and then asserted that the \textit{toXML()} result contained only
the key tag, and that it did not contain the value tag.
\item Next we generated a test response as above with the "resp" type and a test mesage, asserted that the message
was there but that there was no "<Key>" or "<Value>" tags.
\item For reading from a socket, we mocked a socket using mockito, and provided some test XML to put into a byte array
input stream that would be read from in the KVMessage. We then created a KVMessage using the mocked socket and ensured
that it read the key, value and type fields correctly by asserting the return values of the corresponding get methods.
\item We repeat the following with a get request, ensuring that we can get the key, and that the value is null as well
as the message.
\item We also repeat this with a "resp" type message and ensure that we get the correct message string in return and
null for the key and value.
\item Then we try to read an invalid message type from a socket, and ensure that the KVMessage(Socket) constructor
throws a KVException.
\item Finally we create a message to send over the socket (which we are still mocking) this is just a plain string, not
XML, and ensure that when we try to create a KVMessage with this socket an exception is thrown as well because it is not
valid XML.
\end{enumerate}

\section*{Task 2: KVClient}
\subsection*{Correctness Constraints}
\begin{itemize}
\item The client must generate proper XML messages to send to the server.
\item The client must be fault tolerant and throw exceptions when it recieves and error from the server.
\end{itemize}

\subsection*{Declarations}
There are no additional declarations eneded for this section.

\subsection*{Description}
For KVClient we must implement the \textit{connectHost}, \textit{closeHost}, \textit{put}, \textit{get} and \textit{del}
methods.

The \textit{connectHost} method is rather trivial to implement and requires only the creation of a new Socket with the
hostname and port number of the server. The \textit{closeHost} method is rather trivial as well because it takes the
socket that is returned from \textit{connectHost} after we're done with it and calls \textit{close()} on that socket.

The \textit{put}, \textit{get} and \textit{del} methods are a little more complicated but pseudo-code for each of these
is below:
\begin{verbatim}
put(key, value)
    KVMessage message = new KVMessage("putreq");
    message.setKey(key);
    message.setValue(value);
    Socket connection = connectHost();
    message.sendMessage(connection);
    KVMessage response = new KVMessage(connection);
    if(!"Success".equals(response.getMessage())
        throw new KVException(response);

get(key)
    KVMessage message = new KVMessage("getreq");
    message.setKey(key);
    Socket connection = connectHost();
    message.sendMessage(connection);
    KVMessage response = new KVMessage(connection);
    if(!key.equals(response.getKey())
        throw new KVException(response);
    return response.getValue();

del(key)
    KVMessage message = new KVMessage("delreq");
    message.setKey(key);
    Socket connection = connectHost();
    message.sendMessage(connection);
    KVMessage response = new KVMessage(connection);
    if(!"Success".equals(response.getMessage())
        throw new KVException(response);
\end{verbatim}

\subsection*{Testing Plan}

For set-up/tear-down of our tests we would instantiate a SocketServer, KVClientHandler, KVServer and KVClient in order
to use in each of our tests. In the set-up, after instantiating all the objects, we will add the hander to the
SockerServer, and then connect our SocketServer and then spawn off a thread in which it runs. Our KVClient will
connect to the same port that the sever is listening on.

Then for each of our tests our work-flows are as follows:
\begin{enumerate}
\item testPutGet: Use the client to put a test key and value. Then assert that the client get call with that test key
returns the same test value.
\item testDel: Use the client to put a test key and value, then call \textit{del} with that key, then call \textit{get}
with the same key and ensure that it throws a KVException.
\item testFail: Use the client to call \textit{get} on a non-existent key and ensure that a KVException is thrown
and that the message type is a "resp" with message of "Does not exist"
\end{enumerate}

\section*{Task 3: Write-Through KVCache}
\subsection*{Correctness Constraints}
\begin{itemize}
\item Our cache should be set-associative.
\item It should replace entries using the second chance algorithm.
\end{itemize}

\subsection*{Declarations}
\begin{itemize}
\item \textit{ArrayList<LinkedList<CacheNode>> caches(numSets)} - An array of LinkedList<CacheNode> which functions as our set of sub-caches.
\item \textit{WriteLock locks[numSets]} - An array of WriteLocks which restrict/allow access to the given set.
\end{itemize}
We also declare a CacheNode class for keeping track of the elements within our cache. 
Each cache item consists of the key and the value, as well as the reference bit for our second chance policy. Thus we define our CacheNode
class to have those fields: String \textit{key}, String \textit{value}, boolean \textit{reference}.

\subsection*{Description}
Now we need to implement the following methods: the constructor, \textit{get}, \textit{put}, \textit{del}, \textit{getWriteLock}.

The constructor is fairly straight forward to implement. All we need to do is fill our array \textit{caches} with new empty lists and
create corresponding WriteLocks to put in our locks array. The \textit{getWriteLock} method is also fairly simple, all we do is return the
lock from the array corresponding to the index specified by \textit{getSetId} of the key for which we are requesting a write lock.

The get method works as following:
\begin{verbatim}
get(key):
    int setId = getSetId(key);
    for(CacheNode node : caches[setId])
        if(node.key == key)
            node.reference = true;
            return node.key
    return null;
\end{verbatim}

The del method is also quite similar:
\begin{verbatim}
del(key):
    int setId = getSetId(key);
    for(int i = 0; i < caches[setId].size(); i++)
        CacheNode node = caches[setId].get(i);
        if(node.key == key)
            caches[setId].remove(i);
\end{verbatim}

The put method is the most complicated though. In order to ensure second-chance works we must check 
\begin{verbatim}
put(key, value):
    int setId = getSetId(key);
    // Make sure our element isn't already there and update if it is
    for(CacheNode node : caches[setId])
        if(node.key == key)
            node.value = value
            node.reference = true
            updated = true
            return;

    if(caches[setId].size() == maxElemsPerSet)
        // We must proceed with second chance.
        CacheNode oldest = caches[setId].poll()
        while(oldest.referenced)
            oldest.referenced = false // Second chance
            caches[setId].offer(oldest)
            oldest = caches[setId].poll()

    // We've just dropped the oldest from the cache
    // Or there was still room
    caches[setId].offer(new CacheNode(key, value, false))
\end{verbatim} 

\subsection*{Testing Plan}
In order to test our cache, we want to make sure that the elementary put/get/del operations
all work as intended, but also that our replacement policy works as intended. So, in our
set-up method we will initialize a new KVCache for each test with 1 set and 2 elements per set.
Then our tests look as follows:
\begin{enumerate}
\item testPutGet: Call cache.put with a test key/value and then assert that a get call on the same key
returns the correct value.
\item testDel: Puts a key/value in the cache using \textit{put()} then call \textit{del} on the key and then
assert that a \textit{get} call on that key returns null.
\item testOverwrite: Put a key/value into the store, then put the same key with a new value into the store, and
finally assert that a get call returns the new value.
\item testReplacement: Put 3 values into the cache. Since our cache has a capacity of only two, the oldest value
should get pushed out, so assert that calling get on the first value returns null. Then we call put with the second
value, to put it back in, and then add another value. If our referencing policy works correctly, value 2 should have
been referenced so when we put in value 4, value 3 will have to be removed. If we put a 5th value back in, 2 should get
removed because its referenced value should be false due to second chance and so we can assert that after putting 5 into
the cache, 2 is gone. Finally we will call get on 4, to make it referenced and put a value 6 into the cache, which
should force 5 which was never referenced to be evicted.
\end{enumerate}

\section*{Task 4: dumpToFile, restoreToFile, toXML}
\begin{itemize}
\item KVStore is a way to store Key-Value pairs.
\item dumpToFile() puts what is in a KVStore object into the file filename.
\item restoreFromFile() actually replaces the KVStore object with the contents of the file.
\item toXML() converts a KVStore to an XML string
\end{itemize}
\subsection*{Correctness Constraints}
\begin{itemize}
\item Keys are non-null strings with 0 $<$ length $<$ 256 bytes.
\item Values are non-null strings with 0 $<$ length $<$ 256 kilobytes.
\item We will use the built-in javio.io.FileWriter and java.io.BufferedWriter 
classes for dumpToFile in KVStore.

\end{itemize}

\subsection*{Declarations}
There are no new necessary declarations outside of each local declaration inside of each method.

\subsection*{Description}
\begin{itemize}
\item
We will need to import java.io.File, org.w3c.dom.*, org.w3c.dom.NodeList, javax.xml.parsers.*, among many others used as a component API of the Java API for XML Processing.

\item 
toXML()\\
KVStore.toXML() generates an XML string for KVStores. See code below:
\begin{verbatim}
KVStore.toXML(){
  try{
    Document doc = DocumentBuilderFactory.newInstance().newDocumentBuilder().newDocument();
    Element rootElem = xmlDoc.createElement("KVStore");
    doc.appendChild(rootElem);
    iterate through this.store.keys using Map.Entry<String, String>{
      Get the key and value at the ith iteraition.
      Create a childElement = doc.createElement("KVPair");
      childElement.appendChild(doc.createElement("Key").setTextContent(key at ith iteration));
      childElement.appendChild(doc.createElement("Value").setTextContent(value at ith iteration);
      append the childElement to the rootElem
    }
    Create a transformer using TransformerFactory.newInstance.newTransformer() called transformer;
    Create a new StringWriter() called strWriter;
    Create a new StreamResult(strWriter) called streamResult;
    transformer.transform(new DOMSource(doc), streamResult);
    return strWriter.toString();
  }
  catch (ParserConfigurationException e){ // This could be thrown by the newDocumentBuilder() call
    e.printStackTrace(System.out);
  }
  catch(Exception e){
    e.printStackTrace(System.out);
  }
}
\end{verbatim}

\item 
dumpToFile()\\
BufferedWriter is a character streams class. It allows us to write strings directly to the file.
Because the FileWriter constructor and BufferedWriter.write() possibly throw an IOException, we catch those specifically. We also catch all other errors which could arise and then print a stack trace.
\begin{verbatim}
public void dumpToFile(String fileName){
  try{
    Create a new FileWriter(file_name) called file_writer;
    Create a new BufferedWriter(file_writer) called buffered_writer;
    String str = this.toXML();
    buffered_writer.write(str);
    buffered_writer.close();
  }
  catch (IOException e) { // From either BufferedWriter.write() or the
    // FileWriter constructor
    e.printStackTrace(System.out);
  }
  catch (Exception e) {
    e.printStackTrace(System.out);
  }
}
\end{verbatim}

\item restoreFromFile() replaces the KVStore object with the contents of the file using the DocumentBuilder, File, and Document classes.
\begin{verbatim}
public void restoreFromFile(String fileName){
  resetInventory();
  Create a new DocumentBuilder, called builder, using 
    the call DocumentBuilderFactory.newInstance().newDocumentBuilder();
  Create new File(fileName) called file.
  Check if this file exists and is readable (if not, print a statement and return).
  Document d = builder.parse(file);
  Normalize the text representation using d.getDocumentElement().normalize();

  NodeList pairList = d.getElementsByTagName("KVPair");

  For each Node node in pairList{
    get the keyNode using pairList.item(i).getFirstChild();
    get the valueNode using pairList.item(i).getLastChild();
    store these this.store using the call 
      this.store.put(keyNode.getNodeValue(), valueNode.getNodeValue());
  }

  // Caused from the DocumentBuilderFactory.newInstance() call
  catch(FactoryConfigurationError f){
    f.printStackTrace(System.out);
  } catch (IOException i) { // Caused by the DocumentBuilder.parse() call
    i.printStackTrace(System.out);
  } catch (SAXException s) { // Caused by the DocumentBuilder.parse() call
    s.printStackTrace(System.out);
  } catch (Exception e) {
    e.printStackTrace(System.out);
  }
}
  \end{verbatim}

\item 
KVCache.toXML() is essentially the same as KVStore.toXML() except on the first level we iterate over each cache set and wrap all the contents of this within a $<Set>$ tag,
then we iterate over each CacheNode and place the key/value contents within a $<CacheEntry>$ tag and keep note of whether the cache entry has been referenced in the attributes
of that tag. 

\end{itemize}

\subsection*{Testing Plan}

Basically the big thing we want to test with the KVStore dump and restore methods are that the state is persisted to a
file with all of the corresponding information and restored to the store back in the exact same state as previously when
we restore from the file. So, to do this, we create a temp file using JUnit which we can read and write to. Then we
follow the following workflow:
\begin{enumerate}
\item store.put("Maliena", "20");
\item assertEquals("20", store.get("Maliena")); // Just make sure that put/get work, for sanity
\item store.put("Hamel", "19");
\item store.put("Bryan", "24");
\item store.put("EunSeon", "20");
\item store.dumpToFile(dumpFile.getAbsolutePath());
\item store.del("Bryan");
\item store.del("Hamel");
\item store.put("Lisa", "50");
\item store.put("Maliena", "21");
\item store.restoreFromFile(dumpFile.getAbsolutePath());
\item assertFails(KVException, store.get("Lisa")); // Added fields should be removed
\item assertEquals("20", store.get("Maliena")); // Changed fields should be the old value
\item assertEquals("19", store.get("Hamel")); // Removed fields should be restored
\end{enumerate}

\section*{Task 5: ThreadPool}
\subsection*{Correctness Constraints}
\begin{itemize}
\item Our ThreadPool must be thread-safe, in the case of multiple accesses from our WorkerThreads at the same time.
\end{itemize}

\subsection*{Declarations}
In ThreadPool we will need to add the following:
\begin{itemize}
\item \textit{Queue$<Runnable>$ jobQueue} - A queue for holding the list of jobs, ie. a linked list. It will
be @GuardedBy(this).
\end{itemize}
In WorkerThread we will need to add the following:
\begin{itemize}
\item \textit{ThreadPool pool} - A reference to the pool in which this belongs.
\end{itemize}

\subsection*{Description}

For the constructor of ThreadPool which takes a size, we will initialize the array of threads and fill it with the
required number of threads and start them so they wait for jobs to run:
\begin{verbatim}
ThreadPool(size):
    threads[] = new Thread[size];
    for(int i = 0; i < size; i++)
        threads[i] = new WorkerThread(this)
        threads[i].start();
\end{verbatim}

Then for our \textit{addToQueue} method we need to do two things: add our job to the queue safely, and then notify anyone
who is waiting for a job that a new one has been added. In order to make sure that our job is added safely we will synchronize
this method and then we will call \textit{notify()} to let waiting threads know a job has been added.
\begin{verbatim}
synchronized addToQueue(r):
    jobQueue.add(r);
    notify();
\end{verbatim}

Finally for our \textit{getJob} method, which will primarily be called by WorkerThreads, we must make sure that is safely removes
a job from the queue if there is one present or blocks until there is one. This can be done in the following manner:
\begin{verbatim}
synchronized getJob():
    while(jobQueue.peek() == null)
        wait();
    return jobQueue.poll();
\end{verbatim}

Then in the WorkerThread constructor we need to store the provided thread pool in the \textit{pool} variable so we have a reference
to the pool it was created for. After that all we need to do is implement run to get a job from the queue, run it until completion
and then repeat endlessly:
\begin{verbatim}
run():
    while(true)
        Runnable toRun = pool.getJob();
        toRun.run();
\end{verbatim}


\subsection*{Testing Plan}

For testing our ThreadPool we will test two cases, that one thread can complete all the required jobs, and that 10
concurrent threads can complete all the tasks as well. To do this we will create a bogus test job which is a runnable
that just increments an atomic integer and sleeps for 10 milliseconds. Then we test two cases:
\begin{enumerate}
\item testIndividualPool: create a pool with one worker thread, add 100 jobs, sleep for 3 seconds (should be more than
enough time) and then assert that all the jobs were completed by checking that the atomic integer is 100.
\item testBigPool: create a pool with 10 worker threads, add 100 jobs, sleep for 3 seconds (again, this should be more
than enough time) and then assert that the atomic integer has a value of 100.
\item
\end{enumerate}

\section*{Task 6: SocketServer, KVClientHandler, KVServer}
\subsection*{Correctness Constraints}
\begin{itemize}
\item GET(Key k): Retrieves the key-value pair corresponding to the provided key.
\item PUT(Key k, Value v): Inserts the key-value pair into the store.
\item DEL(Key k): Removes the key-value pair corresponding to the provided key from the store.
\item Reads (GETs) and updates (PUTs and DELETEs) are atomic
\item All operations (GETs, PUTs, and DELETEs) must be parallel across different sets in the KVCache, and they cannot be performed in parallel within the same set.
\end{itemize}
\subsection*{Declarations}
\begin{itemize}
\item \textit{SocketServer.running} default initialized to false, tells us whether the server should continue accepting
requests.
\end{itemize}
\subsection*{Description}

\subsubsection*{SocketSever}
For this class, all we need to do is implement the \textit{connect}, \textit{run}, \textit{stop} and
\textit{closeSocket} methods. 

The \textit{connect()} method connects the server to the provided hostname and portname that it was instantiated with.
Thus all we need to do is instantiate our server socket, to listen on the provided port.

The \textit{run()} method opens the server to recieve connections and listens for them. Thus we need to keep
in an endless loop, listening for clients and then handing them off to the handler. It will work in the following
fashion:
\begin{verbatim}
run():
    while(running)
        Socket client = server.accept()
        handler.handle(client)
\end{verbatim}

The \textit{stop()} method stops the server from recieving new connections. This can just be done by setting the
\textit{running} variable to false. That way, after the current connection that is to be handled, the run method will
exit.

The \textit{closeSocket()} method closes and releases the socket that the server has been using to listen for
connections. To do this we just need to call, \textit{close()} on our ServerSocket.

\subsubsection*{KVClientHandler}

This handler is expected to take a socket connection representing a client's request and read that data and
perform the appropriate request to the KVServer. This can be done by creating a KVMessage on that socket to parse the
message sent to the server. Then based on the request being made, we perform one of the KVServer methods of put/get/del.
Depending on whether that request was a success, we create a message to return to the end user and we send it over the
provided socket. This is also where we perform any final bulletproofing that needs to be done. This method should not
throw any exceptions, but rather it should be catching all exceptions and returning a error KVMessage to the end user.

Our \textit{ClientHandler.run()} method looks like the following:
\begin{verbatim}
run():
    KVMessage incoming = new KVMessage(client);
    String type = incoming.getMsgType()
    String key = incoming.getKey()
    String value = incoming.getValue()
    try
        if(type == "putreq")
            server.put(key, value)
        else if(type == "getreq")
            server.get(key)
        else if(type == "delreq")
            server.del(key)
        // We should not hit the else case ever because it's impossible to form a
        // KVMessage with an invalid type, but just in case we do the following
        else
            throw new KVException(new KVMessage("resp", "Unknown error: invalid message type provided"));
        // Since we got this far without an error our message must have been a success
        new KVMessage("resp", "Success").sendMessage(client)
    catch(KVException e)
        e.getMsg().sendMessage(client);
    finally
        client.close()
\end{verbatim}

\subsubsection*{KVServer}

This class delegates all requests for putting/getting keys through the cache and if necessary into the data store.
It is also responsible for ensuring atomicity for example to make sure that a PUT/DEL request aren't executed at the
same time causing problems. The methods which need to be implemented are \textit{put}, \textit{get}, \textit{del}. Also,
since our cache is a write-through cache, we must make sure that our writing methods, not only write to the cache, but
also write to main memory.

For \textit{put} we must put the value into our keystore, also if it is in the cache we must update it in the cache. Our
atomicity requirements state that this must be a fully atomic operation but that it must be able to be run in parallel
across different cache sets. For this to work we must wrap the cache call in the cache lock, but then we must wrap the whole
thing in a synchronized call in order to ensure cache/keystore consistency.
\begin{verbatim}
synchronized put(key, value):
    synchronized(cache.getWriteLock(key)) {
        if(cache.get(key) != null)
            cache.put(key, value)
    }
    keystore.put(key, value)
\end{verbatim}

For \textit{get} we first check to see if the value is in our cache, if so we return it, otherwise we read out of our
keystore and put it into our cache. To do this, we only need to acquire the cache lock, and then if that fails then we
need to access the server so then we wrap the server commands in a synchronized block.
\begin{verbatim}
get(key)
    synchronized(cache.getWriteLock(key)) {
        result = cache.get(key)
        if(result == null)
            synchronized {
                result = server.get(key)
            }
            if(result == null)
                throw new KVException(new KVMessage("resp", "non existent key"))
            else
                cache.put(key, result)
    }
    return result
\end{verbatim}

Finally for \textit{del} we must check to see if the key is in our cache, if so, remove it from the cache, then we must
remove it from the keystore if it is there. If not we throw an exception notifying the user that the key is
non-existent.
\begin{verbatim}
synchronized del(key)
    if(server.get(key) == null)
        throw new KVException(new KVMessage("resp", "non existent key"))
    synchronized(cache.getWriteLock(key)) {
        cache.del(key)
    }
    server.del(key)
\end{verbatim}

\subsection*{Testing Plan}

For this section we will need to write tests for each of our classes. 

For \textbf{SocketServer} we want to ensure that it can handle multiple requests at the same time, this can be done by
having our SocketServer listen on a certain port, then create bunch of bogus request threads that connect to the
listening socket, and send some information. We also create a test handler that handles our bogus requests just to
ensure that they are all passed to the handler. We also want to ensure that the listening and stopping methods work, so
we have the following test cases:
\begin{enumerate}
\item testListening: Here we just create a socket to our server host and port and try to send some data. As long as we
do not get any exceptions we know all is well.
\item testStop: Here we call stop on the server, wait a second, and then try to send some data. The socket should send
the data but it should not be handled so we should get no exceptions, but the handler should show 0 recieved messages.
Then we close the server connection and try to send some data and expect that an exception happens.
\item testMultipleConnections: Here we create a bunch of Threads which send our test message to the server. Then we
spawn several of them off and join them. After they all have completed we assert that the number of recieved messages
by the handler is equal to the number of threads we spawned off.
\end{enumerate}
The during set up and tear-down of each of the tests we create a new SocketServer, add a bogus handler that just counts
the number of time it recieved the message "test" and start the server in a new thread.

For \textbf{KVClientHandler} we need to make sure that each type of request is handled correctly, and that our code is
bulletproofed so any bad information passed into the system is handled gracefully and fails without taking the rest of
the system down. This can be done by passing a variety of sockets into the handler's \textit{handle} method to which we
write a information consisting of valid KVMessages as well as garbage messages and make sure we get the correct results
written back. This leads to the following test cases:
\begin{enumerate}
\item testGoodMessage: Mock a socket connection, create a KVMessage and load it's text into the mock socket, then submit
it to the handler and ensure that it performs the correct operations on the KVServer and that we get a KVMessage
returned.
\item testBadMessage: Mock a socket connection, create a KVMessage which will inevitably lead to an error such a get of
a non-existent key. Then have the handler handle that request and ensure that it handles the failure gracefully and
returns the corresponding failure message.
\item testGarbageMessage: Once again we mock a socket connection and load it with a string that isn't even a valid XML
message. Then we have the handler handle the message and we assert that it handled it gracefully and that it returned
the correct failure message, ie. an XML error message.
\end{enumerate}

For \textbf{KVServer} we will need to write tests which ensure that when we \textit{put} a value to the server, a successive
\textit{get} call recieved the same value and after we \textit{del} that key, a successive \textit{get} call will fail
with a non-existent key error. We will also want to make sure that the values we call \textit{get} on the first time get
added to the cache, and that after we \textit{del} that key, it gets removed from the cache as well. To do this we
needed to create another KVServer constructor that would allow us to supply a KVCache and KVServer ourselves in order
to ensure that the correct behavior took place. After this our test cases look as follows:
\begin{enumerate}
\item testPut: Call server.put with a key and value, assert that key is in the store but not in the cache.
\item testGet: Put a value into the server, assert that a get call returns it and that it is put into the cache
afterwards.
\item testDel: Put a value into the server, call delete on that value, assert that it's not in the cache and that
calling get from the store throws an exception for that value. Then assert that calling get from the server also throws
an exception.
\item testCacheDel: Put a value into the server, call get so it gets put into the cache, call delete on that value and
then ensure that it is also removed from the cache.
\end{enumerate}

\end{document}
