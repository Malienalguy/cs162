\documentclass{article}

\usepackage[top=1in, bottom=1in, left=1in, right=1in]{geometry}
\usepackage{listings}

\begin{document}

\title{Project \#1 Threads}
\author{Hamel Ajay Kothari (de), Maliena Guy (dd), Jack Wilson (dx), Bryan Cote-Chang (ed)}
\date{Section 106, TA: Kevin Klues}
\maketitle

\section*{Task 1: KThread.join()}

\subsection*{Correctness Constraints}
\begin{itemize}
\item When join() is called by a given thread (A) on thread (B). Thread A should be put to sleep and added to thread
B's wait queue.
\item When thread B has finished execution, thread A (and to be safe any other threads on its wait queue) should be
woken up and resume execution.
\item PriorityQueues must work (ie. be able to transfer priority) with thread joins.
\end{itemize}

\subsection*{Declarations}
We will only need one field to be added to our KThread class in order to make this implementation work:
\begin{itemize}
\item \textit{waitingThreads} - A ThreadQueue created using \textit{scheduler.newThreadQueue(true)} for storing
threads waiting for this thread to finish.
\end{itemize}

\subsection*{Description}
To implement our join() call we first assert that the thread is not finished and that
we're not calling join() on ourselves.  After that we add the thread that has called join (the currentthread) 
to that queue by calling waitForAccess() and put it to sleep.

In the finish() method of KThread, we iterate over the queue of threads waiting on our thread and place them on the
ready queue by calling \textit{ready()}. This can be done as follows:
\begin{lstlisting}
KThread nextThread = waitingThreads.nextThread();
while(nextThread != null) {
    nextThread.ready();
    nextThread = waitingThreads.nextThread();
}
\end{lstlisting}

We also need to call \textit{waitingThreads.acquire(this)} in the constructor of our KThread in order to ensure
that our KThread recieving priority donations from the contents of the ThreadQueue.

\subsection*{Testing Plan}
We will test our code by creating a \textit{selfTest()} method in our KThread class which contains some test logic
for thread joining. Then we will test our thread joining in the following way:
\begin{itemize}
\item We can test this code by creating some threads which each print a series of messages, and call thread.join() from their
parents and look at the print statements to ensure that the threads print their messages in the correct order.
\end{itemize}

\section*{Task 2: Condition Variables}

\subsection*{Correctness Constraints}
\begin{itemize}
\item When a KThread calls \textit{sleep()} on the given condition, it must be holding the lock for that condition
and it must give it up and be put to sleep.
\item When another KThread calls \textit{wake()} on the given condition, it must also be holding the corresponding
lock and it must place a sleeping thread on the ready queue.
\item When a thread is woken up after calling KThread.sleep(), it must reacquire the lock which it released.
\item When \textit{wakeAll()} is called all the threads sleeping on the condition must be woken up.
\item The operations should all be completely atomic through the use of interrupts.
\end{itemize}

\subsection*{Declarations}
\begin{itemize}
\item \textit{conditionLock} - The lock which we hold on to when we are using our condition.
\item \textit{waitingThreads} - A ThreadQueue created using \textit{scheduler.newThreadQueue(true)} for holding on
to the threads which are waiting on this condition.
\end{itemize}

\subsection*{Description}
We implement the \textit{sleep()} method as follows:
\begin{enumerate}
\item Disable Interrupts (BEGIN CRITICAL SECTION) 
\item Release the lock 
\item Add our thread to the wait queue 
\item Call KThread.sleep() (ENDS/BEGINS CRITICAL SECTION)
\item Acquire the lock.
\item Restore interrupts (ENDS CRITICAL SECTION) 
\end{enumerate}

Similarly, in the wake() method we do the following: 
\begin{enumerate}
\item Disable interrupts (BEGIN CRITICAL SECTION) 
\item Take an element off the wait queue
\item If the element is not null, call ready to show that the thread is now ready to run.
\item Restore interrupts (BEGIN CRITICAL SECTION)
\end{enumerate}

For the \textit{wakeAll()} method we perform the same logic as in the wake method but rather than taking only
the next thread. We will loop over all the threads while our \textit{nextThread} is not null, we add it to the ready
queue.

\subsection*{Testing Plan}

In order to test our code we can create test lock and test condition. Then we can create a test thread which
acquires the lock, wakes on the test condition, 

\section*{Task 3: Alarm}

\subsection*{Correctness Constraints}
\begin{itemize}
\item When we call Alarm.waitUntil(x) we want our thread to be put to sleep for at least $x$ ticks.
\item On each timerInterrupt we want to make sure that there are no threads sleeping which should have been woken by
the given Machine time.
\end{itemize}

\subsection*{Declarations}
\begin{itemize}
\item \textit{alarmQueue} - A java.util.PriorityQueue which tracks the threads which need to be woken up
and their corresponding times, sorted using a comparator which orders them by their most soon wake time.
\end{itemize}

\subsection*{Description}

In order to implement our Alarm class we need to implement the waitUntil method and the timerInterrupt method. To
make our lives easier we will create a container class called AlarmNode which contains each element we put into our
\textit{alarmQueue} which simply contains our KThread which we have put to sleep and the time at which we want to
wake it up.

Then for our PriorityQueue, we initialize with a comparator which compares based on the wakeTime field of our
AlarmNode class so that whenever we retrieve the next element from our queue, it is the earliest one due to be woken
up.

Now, for the implementation of waitUntil we first calculate the time to wake up our thread as the current system time,
plus the parameter $x$. Then we disable interrupts to ensure that the next operations are atomic. From here we create
an AlarmNode object with the current thread which is calling waitUntil and the wakeTime we calculated earlier. We add
this to the queue, and then put the current thread to sleep. After the KThread.sleep() call we restore interrupts.

For our timerInterrupt we must ensure that all the threads that need to be woken up at this point are woken up. To do
so we can peek() the first element off our queue, and if it's not null and it's wakeTime is less than or equal to
the current machine time meaning that the time that it was supposed to be woken has passed, we take it off the 
queue and call ready. We repeat this until the while loop is false as follows:
\begin{lstlisting}
long machineTime = Machine.timer().getTime();
while((alarmQueue.peek() != null) && (alarmQueue.peek().wakeTime <= machineTime)) {
    alarmQueue.poll().thread.ready();
}
\end{lstlisting}

\subsection*{Testing Plan}

We can easily test that our Alarm class works by creating two threads, each which print the time they start at
(eg. "Test Thread A: Started at 2000"), and then call Alarm.waitUntil with different times, in our case we sleep
for 2000 ticks and 50 ticks, and then after this we print the time they finish at. We then fork off each of these
two threads and join them both and look at the output. In the case we defined above, we should see Thread B finishing
first somewhere in the order of 50 ticks after it is added to the \textit{alarmQueue} after that we should see
Thread A finish somewhere around 2000 ticks after it is added to the queue.

\section*{Task 4: Synchronous Send/Recieve}

\subsection*{Correctness Constraints}
\begin{itemize}
\item A speaker will not terminate unless its message has been recieved by a listener.
\item A listener will not terminate unless it has recieved a message from a speaker.
\item A speaker will communicate a message to one listener, therefore no two listeners can terminate by
recieving the same message.
\end{itemize}

\subsection*{Declarations}
\begin{itemize}
\item \textit{messageLock} - a lock providing protection for our message and listenerCounts.
\item \textit{message} - an integer represent the message we are sending/recieving.
\item \textit{messageInUse} - a boolean represent the message is being used or whether it can be overwritten because
it has been read by a listener.
\item \textit{listenerCount} - a count keeping track of whether there are listeners currently waiting for messages
\item \textit{listeningCondition} - a condition for speakers to sleep on if there are no listeners present.
\item \textit{speakingCondition} - a condition for listeners to sleep on if there is not currently a message to read.
\end{itemize}

\subsection*{Description}
Our send method will work as follows:
\begin{enumerate}
\item Acquire lock
\item Check listenerCount and if listenerCount = 0 or the message is in use, sleep on listeningCondition.  
\item Set our message in use vriable to true.
\item Set our message variable to our desired message. 
\item Wake on spoken condition.
\item Release our lock
\end{enumerate}

Our receive will work as follows:
\begin{enumerate}
\item Acquire lock 
\item listenerCount++ 
\item Wake on listeningCondition 
\item sleep on spokenCondition 
\item Get our message variable
\item Set our message in use variable to false.
\item listenerCount-- 
\item wake any sleeping speakers waiting to send a message
\item Release our lock 
\end{enumerate}

\subsection*{Testing Plan}

We can test our code by having a permutation of speakers and listeners started in different orders and ensuring that
messages are received in the right amount and order. To do this we create 4 threads, one which sends two messages, and
one which sends one message and two threads which listen for messages in a similar manner. We log when each call to
speak() and listen() happens to keep track of when they begin and terminate speaking to ensure that they're terminating
in an acceptable manner. We also look at the results of the listen calls to ensure that they are unique like the
messages that we sent out.

\section*{Task 5: Priority Scheduling}

\subsection*{Correctness Constraints}
\begin{itemize}
\item When a thread waitsForAccess on a given PriorityQueue it should donate its effective priority to the owner of
the queue if the given queue has \textit{transferPriority} set to true.
\item When a thread acquires a given PriorityQueue it should release the current owner of the thread, forcing it to
remove its donations (and calculate its effective priority), and it should gather donations from the contents of the
queue.
\item Items with higher effective priorities should be chosen over items with lower priorities on the same queue and in
the event of equality, the item which has been on the queue longer should be chosen first.
\item When you set the priority of a given thread, its effective priority and the effective priority of the threads it
donates to should be recalculated.
\end{itemize}

\subsection*{Declarations}
Our declarations for our PriorityQueue involve the following:
\begin{itemize}
\item \textit{stateQueue} - a \textit{java.util.PriorityQueue} which holds the ThreadStates of the KThreads which
are waiting to acquire the resource protected with this queue. This queue will be backed with a comparator which
compares our threads ordering in a descending order of \textit{getEffectivePriority()} and if they match up it compares
in an ascending order of the time that they were put into the queue.
\item \textit{blockingThread} - the ThreadState of the KThread which is blocking access to this queue (ie. the state
for the KThread which owns the resource which this queue protects.).
\end{itemize}

For our ThreadState class, our declarations are as follows:
\begin{itemize}
\item \textit{thread} - The KThread which this state object tracks.
\item \textit{priority} - The vanilla priority of our KThread which can be gotten/set by the user.
\item \textit{effectivePriority} - an integer representing the calculated effective priority which is recalculated
when our thread's \textit{priority} changes, or when new threads wait for access to the resources which our thread
owns, or alternatively any of the threads within a queue of our owned resources change priority.
\item \textit{waitingQueue} - A PriorityQueue (not java.util.PriorityQueue) which is a reference to the queue which
our KThread is currently waiting on.
\item \textit{waitingTime} - A long representing the Machine time at the point which our KThread was added to the
\textit{waitingQueue}.
\item \textit{ownedQueue} - A \textit{HashSet<PriorityQueue>} which contains references to all the PriorityQueues
which our KThread has gained ownership of.
\end{itemize}

\subsection*{Description}

For PriorityQueue we need to implement a few methods to get our solution to work. The first method we would logically
implement would be the \textit{pickNextThread()} method. This one is fairly simple using our java PriorityQueue by calling
\textit{stateQueue.peek()} to return the next element in our queue sorted by priority or null if there are no elements
in our queue.

Our next method we must implement is \textit{nextThread()} which is also reasonably simple. We first ensure that our
\textit{!stateQueue.isEmpty()} if it is empty, we return null, otherwise we call \textit{stateQueue.poll()} to get
the next ThreadState. We then call acquire on that next thread state and return the corresponding thread for that thread
state.

The bulk of our logic for our PriorityQueue actually resides in the ThreadStates of our threads. The methods which
we have implemented are \textit{setPriority()}, \textit{getPriority()}, \textit{getEffectivePriority()},
\textit{waitForAccess()}, \textit{acquire()}, \textit{release()} and \textit{recalculateEffectivePriority()}.

The \textit{getPriority()} and \textit{getEffectivePriority()} methods each return the corresponding fields,
\textit{priority} and \textit{effectivePriority}. The \textit{setPriority()} method is reasonably simple as well,
it just sets the \textit{priority} field and then calls \textit{recalculateEffectivePriority()} method delegating
most of its logic there.

Our \textit{waitForAccess()} it a bit more complicated. We want to place our thread on the \textit{stateQueue}
and keep track of the time that it was put on the queue as well as recalculate the effective priority of the owner
of the queue if there is on. We do this as follows:
\begin{lstlisting}
waitForAccess(PriorityQueue waitQueue)
    waitingQueue = waitQueue
    waitingTime = Machine.timer().getTime()
    waitingQueue.stateQueue.add(this)
    if waitingQueue.blockingThread != null
        waitingQueue.blockingThread.recalculateEffectivePriority()
\end{lstlisting}

When we acquire ownership of the resource (either through \textit{nextThread()} or \textit{acquire()} in PriorityQueue)
the \textit{acquire()} method of our ThreadState gets called with the PriorityQueue we are acquiring. To handle this
we release the previous owner from the queue, remove ourself from the \textit{stateQueue} if we're in it, and add this
queue to our owned queues. We also set our thread to be the blockingThread of the queue and zero out our
\textit{waitingQueue} field and waitingTime field. Finally we recalculate our effective priority. The looks like the
following:
\begin{lstlisting}
acquire(PriorityQueue waitQueue)
    if waitQueue.blockingThread != null
        waitQueue.blockingThread.release(waitQueue)
    waitQueue.stateQueue.remove(this)
    waitQueue.blockingThread = this
    ownedQueues.add(waitQueue)
    if waitingQueue = waitQueue
        waitingQueue = null
        waitingTime = 0
    recalculateEffectivePriority()
\end{lstlisting}

The \textit{release()} method is a bit more straight forward. All we have to do is remove queue that is passed in
from our \textit{ownedQueues}, set the \textit{blockingThread} of that queue to null because we no longer own it, and
then recalculate our effective priority to factor in that we no longer have donations from that queue.

Finally, our \textit{recalculateEffectivePriority()} method is where a large amount of our logic takes place. Here we
must update our \textit{effectivePriority} as well as propigate the changes to wherever we may have been donating our
priority. The method will function similar to the following:
\begin{lstlisting}
recalculateEffectivePriority()
    if waitingQueue != null
        waitingQueue.stateQueue.remove(this)
    currPriority = priority
    for PriorityQueue owned in ownedQueues
        TheadState ts = owned.pickNextThread()
        if ts != null
            int tsEffective = ts.getEffectivePriority()
            if tsEffective > currPriority
                currPriority = tsEffective
    bool mustPropigate = ((currPriority != effectivePriority) && (waitingQueue != null))
    effectivePriority = currPriority
    if waitingQueue != null
        waitingQueue.stateQueue.add(this)
    if mustPropigate
        if (waitingQueue.transferPriority && waitingQueue.blockingThread != null)
            waitingQueue.blockingThread.recalculateEffectivePriority()
\end{lstlisting}

\subsection*{Testing Plan}

In order to test our code we can set up a simple scenario of priority donation. We will have 3 priority queues:
\textit{pq1}, \textit{pq2}, \textit{pq3} and 5 threads, \textit{t1}...\textit{t5}. Then we will have
\textit{t1}...\textit{t3} waitForAccess on the three corresponding queues. We will also call acquire on the three
queues using \textit{t2}...\textit{t4}. Now if we set the priority of \textit{t3} to 6 we will expect the effective
priority of \textit{t4} to be 6 as well. If we now set the priority of \textit{t5} to 7 and have it waitForAccess on
\textit{pq1} we will expect \textit{t4} to now have an effective priority of 7 as well. Finally if we call
\textit{nextThread()} on \textit{pq1}, \textit{t4} will no longer be the owner of this resource and no longer be getting
donations from the queue, so we should assert that its effective priority is now 1.

\section*{Task 6: Boat Problem}

\subsection*{Correctness Constraints}
\begin{itemize}
\item Our \textit{begin()} method must terminate with the correct sequence of rowing children/adults to/from Oahu
in order to get everyone from Oahu to Molokai.
\item We can only row one adult to the other shore in a boat.
\item We can row either one or two children to the other shore in a boat but no combination of adults and children.
\item Our threads must function pretty much independently (as individual people) and be able to follow the preprogrammed
logic to get everyone to the other side.
\end{itemize}

\subsection*{Declarations}
All of the members defined below will be defined in our Boat class as static members and initialized in the
\textit{begin} method to ensure they are reset after each run.
\begin{itemize}
\item \textit{bg} - Our BoatGrader which we will be calling to trigger the boat to leave an island.
\item \textit{boat} - A RowBoat object, which represents our boat and the operations which make sense for a boat
in the real world. This class will have to be implemented to make our solution work.
\item \textit{numberOfAdultsOnOahu} - an int representing the number of Adults on Oahu, inititalized to 0
\item \textit{numberOfChildrenOnOahu} - an int representing the number of Children on Oahu, initialized to 0
\item \textit{numberOfAdultsOnMolokai} - an int representing the number of Adults on Molokai, initialized to 0
\item \textit{numberOfChildrenOnMolokai} - an int representing the the number of Children on Molokai, initialized to 0
\item \textit{boatLock} - a Lock which ensures atomicity to all of our operations through ownership of the boat.
\item \textit{boatOnOahu} - a condition to sleep on until the boat arrives on Oahu.
\item \textit{boatOnMolokai} - a condition to sleep on until the boat arrives on Molokai.
\item \textit{okForAdultToLeaveOahu} - a condition to sleep on until an adult is allowed to leave Oahu, in our case we
will let this be the case that there is only one child remaining on Oahu.
\item \textit{okForTwoChildrenToLeaveOahu}
\item \textit{okForOneChildToLeaveOahu}
\item \textit{okForChildToLeaveMolokai}
\item \textit{childDeliveredToOtherSide} - a condition which a Child sleeps on when they are riding to another shore
with another child rowing.
\item \textit{simulationStart} - a condition which all of our Child/Adult threads sleep on until all the threads have
been forked and initialized.
\item \textit{simulationOver} - a condition which our main thread waits on until it is time to exit.
\end{itemize}

\subsection*{Description}

To make our lives easier we will create a few helper classes which will represent the objects in our simulations,
namely a RowBoat, a Child and an Adult.

Our RowBoat will represent the one boat we have to use to row back and forth from Oahu and will be protected by our
boatLock. It will have counters for the number of children and adults in the boat as well as an array tracking the
occupants of the boat and a boolean tracking whether the boat is on Oahu. Then our boat will have a couple of logic
methods. The first method is \textit{rowToOtherSide()} which will check the location of the boat, and row to the other
shore and wake all the threads sleeping on the condition that the boat is on the side it just rowed to. During the
rowing process, the boat will also update the counters of people on either side by subtracting its child/adult count
from the shore it was on and adding it to the shore it was going to. It will also set the location of the occupants
of the boat to the new location of the boat and make the appropriate calls to our boat grader depending on if an
adult/child is rowing the boat and if someone is riding in the boat. We also will have a \textit{load} method
which takes a person and puts them in the boat assuming they can fit. It will throw an exception if they can't but
we can ensure they can always fit by checking the contents of the boat before calling \textit{load}.

We also will have representations of our people with a Child and Adult class which track the location of that person
with an \textit{onOahu} boolean, and each of which implement Runnable so they can be run as threads. In the Adult
\textit{run()} method we will make a call to the AdultItinerary passing \textit{this} to say we want to run with this
Adult. We do the same with Child and ChildItinerary.

Now with these structures/tools in place we just need to come up with the actual logic for the Adult and Child
itineraries and the logic of the begin method.

As I said above, in our \textit{begin()} method we will perform the initializations to all of our static members. All of
our counters should be initialized to 0 and our Conditions should be initialized on our one boat lock. After this our
logic for begin is rather simple. Initialize all our threads, fork them off and then sleep until the simulation is over:
\begin{lstlisting}
for(int i = 0; i < adults; i++)
    new KThread(new Adult()).fork();
for(int i = 0; i < children; i++)
    new KThread(new Child()).fork();
// Ensure our threads get to increment their counters
// This yield shouldn't last long
while((numberOfAdultsOnOahu + numberOfChildrenOnOahu) < (adults + children))
    KThread.yield()
boatLock.acquire();
simulationStart.wakeAll();
simulationOver.sleep();
while(numberOfAdultsOnOahu + numberOfChildrenOnOahu > 0)
    simulationOver.sleep()
boatLock.release()
\end{lstlisting}

As for our AdultItinerary logic, our job is actually pretty simple. The one thing to keep in mind is that our Adults
should now row across to the other shore unless there is only one child left on Oahu with them. This is so that we can
ensure that there is at least one child on the other side, who can row back and pick up the remaining child. If they
left when there were no children on the other side the Adult would row across and have to row back to bring the boat
back. Therefore our Adult itinerary looks like this:
\begin{lstlisting}
boatLock.acquire();
numberOfAdultsOnOahu++; //Increment our Adults on Oahu to start
simulationStart.sleep();
while(numberOfChildrenOnOahu != 1 || !boat.isOnOahu())
    okForAdultToLeaveOahu.sleep()
boat.load(adult);
boat.rowToOtherSide()
if(numberOfChildrenOnOahu + numberOfAdultsOnOahu > 0)
    okForChildToLeaveMolokai.wake()
boatLock.release()
\end{lstlisting}
Since we just left Oahu, it's acceptable for us to query the remaining number of people on Oahu as we do above for the
adults because this is information they could easily obtain as they are leaving.

Our ChildItinerary logic is slightly more complicated. We have to be mindful of 3 cases while we are on Oahu: The case
that we are the only child with no adults, the case we are the only child with remaining adults, and the case that there
are multiple children remaining on Oahu. If we are the only child on Oahu, with no adults, we get in the boat and row to
the other side. If we are the only child with adults still there, we wake an adult telling them it's okToLeaveOahu.
We then sleep until the boat arrives back on Oahu and reevaluate the situation. Finally, in the other case we load the
boat with the two children (the first getting in and then sleeping on \textit{childDeliveredToOtherSide} and the other
getting in and rowing away and then waking on that condition) and send them over. In this same case, once we arrive on
Molokai, we check the number of people on Oahu and if there were people when we left, continue, otherwise we wake on
simulation over and break out of our loop. On the other hand of all this if we are on Molokai, we check to see if there
are people on the other shore (information which can easily be ascertained from the last people to arrive on Molokai)
and if so we load a child into the boat and row across to the other side. Our child logic looks like the following:
\begin{lstlisting}
boatLock.acquire()
while(true)
    if(child.isOnOahu)
        while(!boat.isOnOahu)
            boatOnOahu.sleep()
        if(numChildrenOnOahu == 1 && numAdultsOnOahu == 0)
            boat.load(child)
            boat.rowToOtherSide()
            simulationOver.wake()
            break
        else if(numChildrenOnOahu == 1 && numAdultsOnOahu > 0)
            okForAdultToLeaveOahu.wake()
            boatOnOahu.sleep()
            continue
        else
            if(boat.getChildCount == 1)
                boat.load(child)
                boat.rowToOtherSide()
                childDeliveredToOtherSide.wake()
            else
                boat.load(child)
                childDeliveredToOtherSide.sleep()
            if(numChildrenOnOahu + numAdultsOnOahu == 0)
                simulationOver.wake()
                break;
    else
        while(boat.isOnOahu)
            boatOnMolokai.sleep()
        if(numberOfChildrenOnOahu + numOfAdultsOnOahu > 0)
            boat.load(child)
            boat.rowToOtherSide()
        else
            simulationOver.wake()
            break;
boatLock.release()
\end{lstlisting}
\subsection*{Testing Plan}

In order to test that our logic actually works we can write a \textit{selfTest()} method which calls \textit{begin()}
with various permutations of adults and children and watch the output of the methods to ensure that they follow a valid
combination of BoatGrader calls to row our children/adults back and forth until completion and that it terminates when
there are no more people on Oahu.

\section*{Design Doc Questions}
\begin{enumerate}
\item Priority donation for Semaphores is a tricky subject because for Semaphores the waiting takes place until someone
calls Semaphore.V() and at that point threads are woken up off the queue. The problem with this in order to give the
donations to the thread which calls V() we have to know in advance which thread will call V() and then give it the
donation and remove it when the actual call to V() happens, but we don't really have a way of knowing in advance which
thread will call it, so donations in this sense will be tricky.
\item This solution can work, as long as you make sure that you wait for all the threads to be initialized and their
counters to be incremented before you proceed with any logic. There is a potential for flaws in this approach if you
immediately proceed with logic after you increment because it is possible that not all the threads will increment the
counter before you begin performing conditional operations based off the values of these counters which early on include
insufficient data.
\end{enumerate}

\end{document}
